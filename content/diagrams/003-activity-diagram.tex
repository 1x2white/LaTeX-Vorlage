\begin{tikzpicture}

\tikzstyle{area}=[
	rectangle,
	rounded corners,
	minimum width=4.5cm,
	minimum height=14cm,
	draw=green!50,
	fill=green!10
]

\tikzstyle{state}=[
	rectangle,
	rounded corners,
	text width=3cm,
	minimum width=3cm,
	minimum height=1cm,
	align=center,
	draw=black,
	fill=white
]

\tikzstyle{->}=[
	decoration={
		markings,
		mark=at position 1 with {\arrow[scale=3,>=stealth]{>}}
	},
	postaction={decorate},
	rounded corners=5pt
]

% Äussere Felder
\node (area1) [area] at (0, -6) {
	\begin{minipage}[t][13cm]{4.5cm}
		\textbf{User}
	\end{minipage}
};
\node (area2) [area] at (5, -6) {
	\begin{minipage}[t][13cm]{4.5cm}
	\textbf{GUI}
	\end{minipage}
};

\node (area3) [area] at (10, -6) {
	\begin{minipage}[t][13cm]{4.5cm}
	\textbf{Controller}
	\end{minipage}
};


% Erse Kolonne von Zuständen
\node (a0) [shape=circle, fill=black] at (0, -0.5) {};
\node (a1) [state] at (0, -2) {Software starten};
\node (a2) [state] at (0, -4) {\dq öffnen\dq~klicken};
\node (a3) [state] at (0, -6) {Datei auswählen, \\ \dq OK\dq~klicken};
\node (a4) [state] at (0, -8) {Dropdown: \\Format wählen};
\node (a5) [state] at (0, -10) {\dq Export\dq~klicken};
\node (a6) [state] at (0, -12) {Speicherort wählen, \\\dq OK\dq~klicken};

\draw[->] (a0) to (a1);
\draw[->] (a1) to (a2);
\draw[->] (a2) to (a3);
\draw[->] (a3) to (a4);
\draw[->] (a4) to (a5);
\draw[->] (a5) to (a6);

% Zweite Kolonne
\node (b1) [state] at (5, -2) {Fenster anzeigen};
\node (b2) [state] at (5, -4) {Dateiexplorer anzeigen};
\node (b3) [state] at (5, -6) {Liste d. geladenen Dateien anzeigen};
\node (b4) [state] at (5, -8) {\dq Exportieren\dq -Button anzeigen};
\node (b5) [state] at (5, -10) {Dateiexplorer anzeigen};
\node (b6) [state] at (5, -12) {Erfolgsmeldung anzeigen};
\node (b7) [shape=circle, double, outer sep=0.5pt, fill=black, draw=black] at (5, -14) {};


\draw[->] (a1) to (b1);
\draw[->] (a2) to (b2);
\draw[->] (a3) to (b3);
\draw[->] (a4) to (b4);
\draw[->] (a5) to (b5);
\draw[->] (b6) to (b7);

% Dritte Kolonne
\node (c1) [state] at (10, -7) {Inhalt der Dateien laden};
\node (c2) [state] at (10, -9) {Daten umstrukturieren};
\node (c5) [state] at (10, -12) {Neue Dateien schreiben};

\draw[->] (c2) to (c5);
\draw[->] (a3.south) |- (c1);
\draw[->] (a6) -| (2.5, -11) -- (7.5, -11) |- (c2.west);
\draw[->] (c5) to (b6);
\end{tikzpicture}